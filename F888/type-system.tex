\documentclass{article}
\usepackage{amsfonts}
\usepackage{enumerate}

%opening
\title{Type System for F888}
\author{omnit3a}

\begin{document}

\maketitle

\section{Introduction}
The type system of F888 is based upon the Hindley-Milner type system.

\section{Formal Definition}
\subsection{Variable Access}
\begin{equation}
	\frac
		{x:\sigma \in \Gamma}
		{\Gamma \vdash _{D} x : \sigma}
\end{equation}

\subsection{Application}
\begin{equation}
	\frac 
		{\Gamma \vdash _{D} e_{0}: \tau \rightarrow \tau' \qquad \Gamma \vdash _{D} e_{1} : \tau}
		{\Gamma \vdash _{D} e_{0} e_{1} : \tau'}
\end{equation}

\subsection{Abstraction}
\begin{equation}
	\frac
		{\Gamma, x : \tau \vdash _{D} e : \tau'}
		{\Gamma \vdash _{D} \lambda x.e:\tau \rightarrow \tau'}
\end{equation}

\subsection{Let Bindings}
\begin{equation}
	\frac
		{\Gamma \vdash _{D} e_{0} : \sigma \qquad \Gamma, x:\sigma \vdash e_{1}:\tau}
		{\Gamma \vdash _{D} \textup{ let }x = e_{0} \textup{ in }e_{1}:\tau}
\end{equation}

\subsection{Instantiation}
\begin{equation}
	\frac
		{\Gamma \vdash _{D} e : \sigma' \qquad \sigma' \sqsubseteq \sigma}
		{\Gamma \vdash _{D} e : \sigma}
\end{equation}

\subsection{Generalization}
\begin{equation}
	\frac
		{\Gamma \vdash e : \sigma \qquad \alpha \notin \textup{ free}(\Gamma)}
		{\Gamma \vdash _{D} e:\forall\alpha.\sigma}
\end{equation}
\hfill \break

\section{Types}
\subsection{Monotypes}
An object is a monotype if said object is either a type constant (I.E. An integer, a String), or a type variable (I.E. $\alpha$ or $\beta$), where $\alpha$ or $\beta$ may be substituted for a type constant. Monotypes also include the type of a function application. For example, a function accepting an argument of type $\alpha$, and returning a value of type $\beta$ would be type $\alpha \rightarrow \beta$, thus being a monotype.
\subsection{Polytypes}
An object is a polytype if said object is a monotype, or if the type of said object is contains a variable bound by a type.

\end{document}
